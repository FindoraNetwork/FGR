%%%%%%%%%%%%%%%%%%%%%%%
\section{Block Reward}
%%%%%%%%%%%%%%%%%%%%%%%

\subsection{Logic Diagram}

\subsection{Workload in Findora Code}

\subsection{Workload in Tendermint Code}

Recalling the \LSTINLINE{CreateProposalBlock()} function in Tendermint that we mentioned in \ref{tendermint:CreateProposalBlock},
we can go a step further and add a special rule to it, as follows:

\begin{lstlisting}[language=go]
    /**
     * tendermint(v0.33.5)/state/execution.go#92
     */
    func (blockExec *BlockExecutor) CreateProposalBlock(
        height int64,
        state State, commit *types.Commit,
        proposerAddr []byte,
    ) (*types.Block, *types.PartSet) {
        maxBytes := state.ConsensusParams.Block.MaxBytes
        maxGas := state.ConsensusParams.Block.MaxGas

        maxNumEvidence, _ := types.MaxEvidencePerBlock(maxBytes)
        evidence := blockExec.evpool.PendingEvidence(maxNumEvidence)

        maxDataBytes := types.MaxDataBytes(maxBytes, state.Validators.Size(), len(evidence))
        txs := blockExec.mempool.ReapMaxBytesMaxGas(maxDataBytes, maxGas)

        // Right here!
        //
        // Below is the form of `AppendFeeTx()`' in the 'Transaction Fee' section:
        //
        // ```
        // txs = findora.AppendFeeTx(txs)
        // ```
        //
        // Now, suppose our block reward is 10 FRA, consider the following rules:
        //
        // - allow the output in this transaction to be 10 FRA more than the input
        // - the input address must be the address of `PUB_ACCOUNT`
        // - transfer from the `PUB_ACCOUNT` is only allowed once in each block
        //     - prevent verifiers from doing bad things
        //     - ensure this in the ABCI's `DeliverTx()` based on our APP logic
        //
        // That's it!
        // The `transaction fee` and `block rewards` will be done at the same time.
        // Finally, we rename the original `AppendFeeTx()` to `AppendFeeAndRewards()`:
        //
        // ```
        // txs = findora.AppendFeeAndRewards(txs)
        // ```

        return state.MakeBlock(height, txs, commit, evidence, proposerAddr)
    }
\end{lstlisting}
