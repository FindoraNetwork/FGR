\clearpage

%%%%%%%%%%%%%%%%%%%%%%%
\section{Genesis State}
%%%%%%%%%%%%%%%%%%%%%%%

\subsection{Overview}

~\par

Whether it is by adding \LSTINLINE{AppState}\footnote{and then sending the transaction accordingly in the APP}
in \LSTINLINE{genesis.json},
or the way of sending transactions directly on the client side, there has no guarantee that all the preset
transactions will be packaged into the genesis block\footnote{Tendermint's genesis block height is 1}.
The reason is that all transactions sent from outside need to go through tendermint's \LSTINLINE{MemPool},
The time when these transactions in \LSTINLINE{MemPool} are confirmed\footnote{aka `packaged by validator'} cannot be accurately predicted.

If we can add these preset transactions after \LSTINLINE{MemPool}, we will get a definite result,
that is, all preset transactions will appear in the genesis block.

The transaction process in the genesis block is as follows:

\begin{ENUMERATE}
    \item Create an FRA asset type with a fixed address\footnote {this address is similar to FFA and is also fully public}.
            This address will be referred to as FGA (Findora Genesis Address)
            \footnote{Similar to the concept of coinbase in other blockchains} in the following content,
    \item Issue a large number of tokens, such as the maximum value of u64.
    \item Create a transaction, send the corresponding amount of tokens to those pre-registered address.
    \item These transactions will be obtained by the APP through the regular
            \LSTINLINE{DeliverTx} callback, so no additional data synchronization operations are required.
\end{ENUMERATE}

\subsection{Safety Analysis}

\begin{ENUMERATE}
    \item Why issue such a large number of tokens?
    \begin{ITEMIZE}
        \item It can be issued on demand.
        \item The currency issuance strategy here is an alternative.
    \end{ITEMIZE}
    \item Why does FGA need to be fully disclosed?
    \begin{ITEMIZE}
        \item If the private key is kept secret, it will bring a crisis of public trust.
        \item System will ensure that this address is used correctly through a consensus mechanism.
    \end{ITEMIZE}
    \item Why do I need to use FFA when transferring money to users?
    \begin{ITEMIZE}
        \item FGA will be restricted to only allow transfers to FFA in the logic of the APP layer.
        \item This attribute will also be used by the block rewards in the \ref{section:reward} chapter.
    \end{ITEMIZE}
\end{ENUMERATE}

\clearpage

\subsection{Changes to Tenderimint}

~\par

When the block height is 1, it means that it is in the process of publishing the genesis block.

Similar to the previous section, the \LSTINLINE{golang package} named \LSTINLINE{findora} is also required,
The function of \LSTINLINE{txn_builder} is internally encapsulated and used to generate all genesis transactions.

\begin{lstlisting}[language=go]
    /**
     * tendermint(v0.33.5)/state/state.go#131
     */
    func (state State) MakeBlock(
        height int64,
        txs []types.Tx,
        commit *types.Commit,
        evidence []types.Evidence,
        proposerAddress []byte,
    ) (*types.Block, *types.PartSet) {
        block := types.MakeBlock(height, txs, commit, evidence)

        var timestamp time.Time
        if height == 1 {
            // Right here!
            // Write all transactions that need to be in the genesis block.
            //
            // ```
            // txs = findora.GenesisTxs()
            // block = types.MakeBlock(height, txs, commit, evidence)
            // ```
            timestamp = state.LastBlockTime
        } else {
            timestamp = MedianTime(commit, state.LastValidators)
        }

        block.Header.Populate(
            state.Version.Consensus, state.ChainID,
            timestamp, state.LastBlockID,
            state.Validators.Hash(), state.NextValidators.Hash(),
            state.ConsensusParams.Hash(), state.AppHash, state.LastResultsHash,
            proposerAddress,
        )

        return block, block.MakePartSet(types.BlockPartSizeBytes)
    }
\end{lstlisting}
