%%%%%%%%%%%%%%%%%%%%%%%
\section{Transaction Fee}
%%%%%%%%%%%%%%%%%%%%%%%

\subsection{Logic Diagram}

\subsection{Workload in Findora Code}

\subsection{Workload in Tenderimint Code} \label{tendermint:CreateProposalBlock}

\begin{lstlisting}[language=go]
    /**
     * tendermint(v0.33.5)/state/execution.go#92
     */
    func (blockExec *BlockExecutor) CreateProposalBlock(
        height int64,
        state State, commit *tGypes.Commit,
        proposerAddr []byte,
    ) (*types.Block, *types.PartSet) {
        maxBytes := state.ConsensusParams.Block.MaxBytes
        maxGas := state.ConsensusParams.Block.MaxGas

        maxNumEvidence, _ := types.MaxEvidencePerBlock(maxBytes)
        evidence := blockExec.evpool.PendingEvidence(maxNumEvidence)

        maxDataBytes := types.MaxDataBytes(maxBytes, state.Validators.Size(), len(evidence))
        txs := blockExec.mempool.ReapMaxBytesMaxGas(maxDataBytes, maxGas)

        // Right here!
        //
        // Filter out transactions that want to steal fees,
        // drop all txs with any input from the `PUB_ACCOUNT`.
        //
        // Should this be implemented in the ABCI's `CheckTx(...)`?
        //
        // ```
        // txs = findora.FilterThieves(txs)
        // ```
        //
        // After all txs from the mempool have been handled correctly,
        // the validator can create a new tx which will transfer all fees
        // from the `PUB_ACCOUNT` to its own, and append it to txs.
        //
        // ```
        // txs = findora.AppendFeeTx(txs)
        // ```

        return state.MakeBlock(height, txs, commit, evidence, proposerAddr)
    }
\end{lstlisting}
