\clearpage

%%%%%%%%%%%%%%%%%%%%%%%
\section{Transaction Fee}
%%%%%%%%%%%%%%%%%%%%%%%

\subsection{Overview}

~\par

When the client generates a transaction, the receiving account of the transaction fee
is a fixed address, which is completely public, including its private key.
Therefore, the client does not need a special method to set transaction fees,
and all fees transferred to this address are regarded as transaction fees.

In the following content, FFA (Findora Fee Address) will be used to represent this address.

When the Validator generates a block, it can transfer all assets
under the FFA to its own account based on the FFA's private key as the block's revenue, usually FRA tokens.

The calculation rule of transaction fee is another issue that needs to be considered,
such as charging according to the total number of bytes of the transaction itself, etc.
This will be implemented as a method of \LSTINLINE{Transaction} in the code,
which will be referred to in the form of \LSTINLINE{tx.calculate_fee()} later.

The transaction fee collection logic will be implemented in this way.

\subsection{Safety Analysis}

\begin{enumerate}
    \item In a fully public FFA account, anyone can construct a legal transaction signature.
            How to ensure that the tokens are only obtained by the Validator?
        \begin{itemize}
            \item Analyzing from the economic model, Validator will certainly not allow the theft of its assets.
            \item The block is packaged by the Validator. In the process of packaging,
                    all records of transfers from the FFA account that are not issued by itself will be filtered out.
            \item The logic of this part will be implemented by changing the code of Tendermint.
        \end{itemize}
    \item Does Validator have a chance to create double-spending phenomenon?
        \begin{itemize}
            \item This needs to be controlled by the business logic of the APP, such as Nonce, etc.
            \item The transaction fee mechanism itself does not introduce additional double spending.
        \end{itemize}
\end{enumerate}

\subsection{Jobs in Findora}

\begin{itemize}
    \item Provide a function to the client to calculate transaction fees
            during the transaction construction phase.
    \item Ensure that at least one \LSTINLINE{Operation} of type \LSTINLINE{TransferAsset}
            exists in each transaction, and the token type of operation is FRA.
    \item According to the result of \LSTINLINE{tx.calculate_fee()}
            to calculate whether the minimum transaction fee is met.
\end{itemize}

\begin{lstlisting}
    pub struct Transaction {
        pub body: TransactionBody,
        ...
    }

    pub struct TransactionBody {
        // Some `Operation`s in the `TransferAsset` type must exist,
        // and the destination of their FRAs must be the `FFA`.
        pub operations: Vec<Operation>,
        ...
    }

    pub enum Operation {
        TransferAsset(TransferAsset),
        ...
    }
\end{lstlisting}

\subsection{Jobs in Tenderimint} \label{tendermint:CreateProposalBlock}

~\par

Create a \LSTINLINE{golang package} similar to \LSTINLINE{github.com/FindoraNetwork/findora},
The internal package \LSTINLINE{txn_builder}\footnote{ compiles Rust code into a static library,
and Go calls the function through FFI},
It is used to create a transaction for charging transaction fees
when \LSTINLINE{Tendermint} generates a block.

The comment section in the following code segment describes the logic
that will be added to \LSTINLINE{Tendermint}:

\begin{lstlisting}[language=go]
    /**
     * tendermint(v0.33.5)/state/execution.go#92
     */
    func (blockExec *BlockExecutor) CreateProposalBlock(
        height int64,
        state State, commit *tGypes.Commit,
        proposerAddr []byte,
    ) (*types.Block, *types.PartSet) {
        maxBytes := state.ConsensusParams.Block.MaxBytes
        maxGas := state.ConsensusParams.Block.MaxGas

        maxNumEvidence, _ := types.MaxEvidencePerBlock(maxBytes)
        evidence := blockExec.evpool.PendingEvidence(maxNumEvidence)

        maxDataBytes := types.MaxDataBytes(maxBytes, state.Validators.Size(), len(evidence))
        txs := blockExec.mempool.ReapMaxBytesMaxGas(maxDataBytes, maxGas)

        // Right here!
        //
        // Filter out transactions that want to steal fees,
        // drop all txs with any input from the `FFA`.
        //
        // Should this be implemented in the ABCI's `CheckTx(...)`?
        //
        // ```
        // txs = findora.FilterThieves(txs)
        // ```
        //
        // After all txs from the mempool have been handled correctly,
        // the validator can create a new tx which will transfer all fees
        // from the `FFA` to its own, and append it to txs.
        //
        // ```
        // txs = findora.AppendFeeTx(txs)
        // ```

        return state.MakeBlock(height, txs, commit, evidence, proposerAddr)
    }
\end{lstlisting}
