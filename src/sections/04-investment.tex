%%%%%%%%%%%%%%%%%%%%%%%
\section{Investment Return}
%%%%%%%%%%%%%%%%%%%%%%%

\subsection{Overview}

~\par

Some thoughts about the design of the return on investment:

\begin{ITEMIZE}
    \item We will assume that the token will increase in value over time, which is a natural benefit.
    \item During mainnet v1.0, all validators may be operated internally, but this does not mean that the income
        receiving address of the code validator must also be internal. It can be the address of the investor,
        and a single validator can have multiple such addresses. The weight of each address is allocated based
        on the investment amount. In general, validator can perform the role of a revenue distribution agent.
    \item When issuing FRAs, we issue such an investment incentive strategy:
        at a predetermined block height, we will add up all the transaction fees before this
        (maybe use other better algorithms instead of simple addition), Issuance of the same amount
        of tokens as investment income. The method of distribution is: the weight is calculated according
        to the FRA holding time of all investors and the holding amount at that time,
        and the final return that each investor will get is determined according to this weight.
        The principle of this model is: the more transactions, the better the operating status of Findora,
        and the better the operating status of Findora, the greater the expected return of investors,
        which in turn will give investors the motivation to do more things that benefit Findora.
\end{ITEMIZE}